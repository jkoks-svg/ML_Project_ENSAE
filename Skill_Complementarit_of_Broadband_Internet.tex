%%%%%%%%%%%%%%%%%%%%%%%%%%%%%%%%%%%%%%%%%%%%%%
% Documentation
%%%%%%%%%%%%%%%%%%%%%%%%%%%%%%%%%%%%%%%%%%%%%%
\documentclass[compress, aspectratio=169, xcolor=dvipsnames]{beamer}
\usetheme{Madrid}
\useoutertheme[subsection=false]{miniframes} % Alternatively: miniframes, infolines, split
\useinnertheme{circles}

\usepackage{etoolbox}
\makeatletter
\patchcmd{\slideentry}{\advance\beamer@tempdim by -.05cm}{\advance\beamer@tempdim by\beamer@vboxoffset\advance\beamer@tempdim by\beamer@boxsize\advance\beamer@tempdim by 1.2\pgflinewidth}{}{}
\patchcmd{\slideentry}{\kern\beamer@tempdim}{\advance\beamer@tempdim by 2pt\advance\beamer@tempdim by\wd\beamer@sectionbox\kern\beamer@tempdim}{}{}
\makeatother

\definecolor{UBCblue}{rgb}{0.04706, 0.13725, 0.26667} % UBC Blue (primary)
\definecolor{UBCgrey}{rgb}{0.3686, 0.5255, 0.6235} % UBC Grey (secondary)

%\definecolor{UBCblue}{rgb}{0.612, 0.188, 0.227} %ENSAE Color
%\definecolor{UBCgrey}{rgb}{0.851, 0.459, 0.494}

\setbeamercolor{palette primary}{bg=UBCblue,fg=white}
\setbeamercolor{palette secondary}{bg=UBCblue,fg=white}
\setbeamercolor{palette tertiary}{bg=UBCblue,fg=white}
\setbeamercolor{palette quaternary}{bg=UBCgrey,fg=white}
\setbeamercolor{structure}{fg=UBCblue} % itemize, enumerate, etc
\setbeamercolor{section in toc}{fg=UBCblue} % TOC sections
% Override palette coloring with secondary
\setbeamercolor{subsection in head/foot}{bg=UBCgrey,fg=white}
\setbeamertemplate{blocks}[default]
\setbeamertemplate{title page}[default][colsep=-4bp,rounded=true]

\usefonttheme{serif}

\setlength{\parindent}{0pt}
\usepackage[utf8]{inputenc}
\usepackage{amsbsy}
\usepackage{amsmath}
\usepackage{enumitem}
\usepackage{hyperref}
\usepackage[T1]{fontenc}
\usepackage{tikz}
\usepackage{latexsym,xcolor,multicol,booktabs,calligra}
\usepackage{amsmath,amssymb,BOONDOX-cal,bm}	
\usepackage{graphicx,stackengine}   
\usepackage{xcolor}

\beamertemplatenavigationsymbolsempty

\makeatletter
\setbeamertemplate{footline}{
    \leavevmode%
    \hbox{%
        \begin{beamercolorbox}[wd=.3\paperwidth,ht=2.25ex,dp=1.125ex,leftskip=.3cm plus1fill,rightskip=.3cm]{author in head/foot}%
            \usebeamerfont{author in head/foot}\insertshortauthor
        \end{beamercolorbox}%
        \begin{beamercolorbox}[wd=.55\paperwidth,ht=2.25ex,dp=1.125ex,leftskip=.3cm,rightskip=.3cm plus1fil]{title in head/foot}%
            \usebeamerfont{title in head/foot}\insertshorttitle
        \end{beamercolorbox}%
        \begin{beamercolorbox}[wd=.15\paperwidth,ht=2.25ex,dp=1.125ex,leftskip=0cm plus1fill,rightskip=.3cm]{date in head/foot}%
            \usebeamerfont{date in head/foot} \insertframenumber{} / \inserttotalframenumber\hspace*{2ex} 
        \end{beamercolorbox}%
    }%
    \vskip0pt%
}
\makeatother

% Quotation Styles

\usepackage[style=british]{csquotes}

\def\signed #1{{\leavevmode\unskip\nobreak\hfil\penalty50\hskip1em
  \hbox{}\nobreak\hfill #1%
  \parfillskip=0pt \finalhyphendemerits=0 \endgraf}}

\newsavebox\mybox
\newenvironment{aquote}[1]
  {\savebox\mybox{#1}\begin{quote}\openautoquote\hspace*{-.7ex}}
  {\unskip\closeautoquote\vspace*{1mm}\signed{\usebox\mybox}\end{quote}}

%%%%% Presentation Main Page %%%%%

\author{Jakob Sarrazin}

\title[The Skill Complementarity of Broadband Internet]{The Skill Complementarity of Broadband Internet}

\subtitle{{Article by Anders Akerman, Ingvil Gaarder, Magne Mogstad \\ \textit{in The Quarterly Journal of Economics}, July 2015}}

\institute{Michelangelo Rossi \\ Digital Economics \\ ENSAE, Institut Polytechnique de Paris} 

\date{19/02/2025}

\begin{document}

%%%%% Titel and Table of Contents %%%%%%%
	
	\begin{frame}[plain]
    	\titlepage
 	\end{frame}
 	

\addtocounter{framenumber}{-2}
 	\begin{frame}
 	\large{\textbf{Table of Contents}}
		\tableofcontents[sectionstyle=show,subsectionstyle=shaded]
	\end{frame}
	
	
%%%%%%% ------------- ------------- ------------- ------------- %%%%%%%
%%%%%%% Motivation 
%%%%%%% ------------- ------------- ------------- ------------- %%%%%%%

\section{Motivation}
\begin{frame}{Motivation}

\textbf{\underline{Key Topics:} ICT, Policy Intervention, Internet Adoption, Productivity} \newline

\begin{columns}
\begin{column}{0.68\textwidth}
	

\textbf{How does ICT adoption drive productivity?}


\begin{itemize}
	\item [$\rightarrow$] A Major shift in ICT was the transition to broadband connectivity internet.
	\item [$\rightarrow$] Many government agencies fund broadband internet to enhance productivity.
	\item [$\rightarrow$] Missing Research on how high-speed internet adoption affected productivity.
	\end{itemize}
	
\end{column}

\begin{column}{0.3\textwidth}
	\begin{center}
     \includegraphics[width=0.75\textwidth]{www.png}
     \end{center}
\end{column}

\end{columns}


\end{frame}

\begin{frame}{Motivation}
\begin{block}{Research Question}
	How does broadband internet adoption impact labor market outcomes and productivity across different skill levels, and what are the underlying mechanisms driving this effect?
\end{block}
	
\end{frame}

\begin{frame}{Long Story Short}

\textbf{Contribution to Literature} 
\begin{itemize}
	\item[$-$] Builds and extends on the literature on the labor market effects of ICT
	\item[$-$] Novel evidence on the skill bias of broadband internet
\end{itemize}

\vspace{10pt}
\textbf{Main findings} \\
\vspace{3pt}
Adoption of broadband internet $\dots$
\begin{itemize}
	\item[$\dots$] benefits (harms) skilled (unskilled) workers.
	\item[$\dots$] increases firm productivity but distributes the gains unequally among workers.
	\item[$\dots$] is an endogenous technology adoption.
\end{itemize}
	
\end{frame}

%%%%%%% ------------- ------------- ------------- ------------- %%%%%%%
%%%%%%% Methodology 
%%%%%%% ------------- ------------- ------------- ------------- %%%%%%%

\section{Methodology}

\begin{frame}{Context}


\begin{columns}
\begin{column}{0.6\textwidth}

\textbf{Norway}
\begin{itemize}
	\item[$-$] Small open Economy
	\item[$-$] Segmented local labor markets
	\item[$-$] Large and sparsely populated
\end{itemize}

\vspace{10pt}
\textbf{National Broadband Policy}
\begin{itemize}
	\item[$\rightarrow$] \underline{Main Goal:} Ensure supply of broadband internet to every area of the country at a uniform price
	\item[$\rightarrow$] \underline{Second Goal:} Ensure that the public sector quickly adopted broadband internet

\end{itemize}
	
\end{column}

\begin{column}{0.3\textwidth}  %%<--- here
    \begin{center}
     \includegraphics[width=1\textwidth]{Norway.svg.png}
     
     \scriptsize{\textit{source: \\ Wikipedia}}
     \end{center}
\end{column}
\end{columns}
	
\end{frame}

\begin{frame}{Evolution of Broadband Availability}
	\begin{center}
     \includegraphics[width=0.75\textwidth]{Evolution.pdf}
     \end{center}
     \vspace{-10pt}
     \begin{itemize}
     	\item[$\Rightarrow$] While in 2000 only major cities were connected to broadband internet, \\by 2005, most municipalities achieved fairly high rates
     \end{itemize}
      
\end{frame}

\begin{frame}{Data}
	
	\begin{block}{\textbf{Firm and Worker Data} \textit{(Norwegian Tax Authority)}} 
		\begin{itemize}
			\item[$\rightarrow$] \textbf{Firm data} of all nonfinancial joint-stock firms ($\sim 150 K$) \textit{(2000-2006)} \\
			\textit{revenues, capital, labor, intermediates, industry code, municipality level}
			\item[$\rightarrow$] \textbf{Employee data} of all firms and workers \textit{(2000-2008)} \\
			\textit{length of educ. (high/low skilled), annual labor income, approx. hourly wages}
			\end{itemize}
	\end{block}
	\begin{block}{\textbf{Internet Data} \textit{(Statistics Norway and the National Government)}} 
	\begin{itemize}
		\item[$\rightarrow$] \textbf{Broadband subscription} from a random sample of firms
		\item[$\rightarrow$] \textbf{Broadband availability} (independent of takeup)
	\end{itemize}
	\end{block}
	$\rightarrow$ \textit{All Datasets were linked through unique identfiers}

\end{frame}

\begin{frame}{Key Identification Challenge}
\textbf{Main Assumption:} Government broadband rollout program is an exogenous shock. \\
\vspace{10pt}
\textbf{Main concern:} \underline{Timing} of broadband rollout might \underline{correlate} with pre-existing \underline{economic} \underline{trends}, affecting wages or productivity independently of broadband adoption. \\


	\begin{itemize}
		\item[$\rightarrow$] \underline{Test} whether the \underline{expansion} of broadband is \underline{related to pre-existing} municipality \underline{characteristics} \textit{(e.g. Demographics, Economic conditions, Industry structure, Skill composition)}
	\end{itemize}
	\vspace{10pt}
	\textbf{Findings:} Timing of broadband rollout is not correlated with these variables.
	\begin{itemize}
		\item[$\rightarrow$] Supports the exogeneity assumption
		\item[$\rightarrow$]\textit{Only pattern: broadband expansion is correlated with urbanization until 2002}
	\end{itemize}
	
	
\end{frame}


\begin{frame}{Regression Model of Intention-to-Treat Effects}
	\textbf{Panel Data Regression}
	\vspace{-7pt}
	\begin{equation*}
		y_{imt} = x'_{imt} \delta_0 + z_{mt} x'_{imt} \delta_1 + w'_{imt} \theta + \eta_m + \tau_t + u_{imt}
	\end{equation*}
	\vspace{-32pt}
	\begin{center}
	    {\scriptsize \textit{i: firm/individual, m: municipality, t: period}}
	\end{center}
	\begin{small}
	\vspace{-5pt}
	\textit{where:
	\vspace{-3pt}
	\begin{itemize}[noitemsep]
		\item[$-$] $z_{mt}$: availability rate of broadband internet (continous)
		\item[$-$] $w_{imt}$: control variables, industry code
		\item[$-$] $\eta_m$, $\tau_t$: municipality and time fixed effects
	\end{itemize}
	}
	
	\vspace{5pt}
	\textbf{\underline{Use the model to estimate:}}
	\vspace{5pt}
	\begin{columns}[t]
	\begin{column}{0.49\textwidth}
	\textbf{Labor market outcomes}
	\vspace{-3pt}
	\begin{itemize}[noitemsep]
	\item[$-$] $y_{imt}$: (log) hourly wage
	\item[$-$] $x_{imt}$: educational attainment
	\item[$-$] $\delta_1$: interaction between education and broadband availability
	\end{itemize}
	\end{column}
	\begin{column}{0.49\textwidth}
	\textbf{Productivity of high- and low-skilled}
	\vspace{-3pt}
	\begin{itemize}[noitemsep]
	\item[$-$] $y_{imt}$: (log) value added of firm
	\item[$-$] $x_{imt}$: (log) input factors (capital, labor)
	\item[$-$] $\delta_1$: interaction between inputs and broadband availability
	\end{itemize}	
	\end{column}
	\end{columns}

	\end{small}
	

\end{frame}



%%%%%%% ------------- ------------- ------------- ------------- %%%%%%%
%%%%%%% Results 
%%%%%%% ------------- ------------- ------------- ------------- %%%%%%%
\section{Results}




\begin{frame}{Worker-Level Evidence I}

\begin{columns}
	
\begin{column}{0.5\textwidth}
\textbf{Significant Results:}
\begin{itemize}
	\item[$\rightarrow$] 10\%-point $\nearrow$ in broadband availability $\rightarrow$ 0.2\% $\nearrow$ in wages and employment of skilled workers
	\item[$\rightarrow$] No effect for unskilled workers
\end{itemize}
\vspace{5pt}
\textbf{Sum up:}
\begin{itemize}
	\item[$\Rightarrow$] Increased availability of broadband internet \textit{modestly improves the labor market outcomes} of skilled individuals.
\end{itemize}


\end{column}

\begin{column}{0.45\textwidth}
	
	\scriptsize{
	\begin{tabular}{l p{6cm}}
 	\underline{Table 1:} & ITT Effects \newline \textit{on Wages and Employment}
 	\end{tabular}}
 	\centering
    \includegraphics[width=1\textwidth]{TableIII.png}\qquad
    \tiny
    \textit{High Skilled $\Rightarrow$ college degree or higher, else Unskilled}
    
\end{column}
	
\end{columns}
	
\end{frame}

\begin{frame}{Worker-Level Evidence II - Graphical Illustration}

\begin{columns}[t]
	
\begin{column}{0.55\textwidth}

\vspace{5pt}
\textbf{Interpretation:}
\begin{itemize}
	
	\item[$\rightarrow$] In 2007, wages were 1.8\% higher for skilled workers than they would have been in the absence of the technological shock
	\item[$\rightarrow$] Wages for unskilled workers declined by 0.6\%
\end{itemize}
\vspace{5pt}
\textbf{Sum up:}
\begin{itemize}
	\item[$\Rightarrow$] Expansion of broadband internet contributes to an \textit{increase over time in the relative wage} bill share of skilled workers.
\end{itemize}
\end{column}

\begin{column}{0.45\textwidth}
    \centering 
	\vspace{-2pt}
    \scriptsize{
    \begin{tabular}{l p{4.5cm}}
    \underline{Figure 1:} & Actual and Counterfactual Trends in Labor Market Outcomes
    \end{tabular}
    }
    \includegraphics[width=0.7\textwidth]{FigureIII.png}

    
\end{column}
	
\end{columns}
	
\end{frame}


\begin{frame}{Firm Level Evidence}

\begin{columns}
	
\begin{column}{0.55\textwidth}
Try to estimate to which extend increased availability of broadband raised firm productivity \\ 
\vspace{5pt}
\textbf{Significant Results:}
\begin{itemize}
	\item[$\rightarrow$] 10\%-point $\nearrow$ in broadband availability $\rightarrow$ 0.75\% $\nearrow$ in output by high skilled workers
	\item[$\rightarrow$] No effect for output elasticity of capital
\end{itemize}
\vspace{5pt}
\textbf{Sum up:}
\begin{itemize}
	\item[$\Rightarrow$] Increased availability of broadband internet is associated with a \textit{substantial increase in output elasticity} of skilled labor.
	\item[$\Rightarrow$] Firms earn substantial rent from the expansion of broadband internet, low pass-through
\end{itemize}


\end{column}

\begin{column}{0.45\textwidth}
	\centering
	\scriptsize{
	\begin{tabular}{l p{5cm}}
 	\underline{Table 2:} & ITT Effects \textit{on Output Elasticities}
 	\end{tabular}}
 	
    \includegraphics[width=0.85\textwidth]{TableIV.png}\qquad
    \tiny
    \textit{High Skilled $\Rightarrow$ college degree or higher, else Unskilled}
    
\end{column}
	
\end{columns}
	
\end{frame}

\begin{frame}{Broadband Adoption in Firms}

\begin{columns}

\begin{column}{0.6\textwidth}

	\textbf{Model to estimate broadband availability against adoption}
\begin{equation*}
	D_{imt} = \delta z_{mt} + w'_{imt} \theta + \gamma_m + \sigma_t + v_{imt}
\end{equation*}
\begin{itemize}
	\item[$\rightarrow$] Survey Sample of Firms 
	\item[$\rightarrow$] $\hat \delta$: Coefficient on the availability rate
\end{itemize}
\vspace{5pt}
\textbf{Results}
\begin{itemize}
	\item[$\rightarrow$] Strong impact on adoption due to an increase in availability in the previous year: 10\%-point availability $\nearrow$ $\rightarrow$ +2.3\% firms adopt internet
\end{itemize}

\end{column}

\begin{column}{0.4\textwidth}
	\centering
	\scriptsize{
	\begin{tabular}{l p{4cm}}
 	\underline{Figure 2:} & Association between Availability and Usage Rates
 	\end{tabular}}
 	\vspace{10pt}
 	
    \includegraphics[width=1\textwidth]{FigureIV.png}\qquad
    
\end{column}


\end{columns}	

\end{frame}

\begin{frame}{Understand the Pattern of Adoption}

\begin{columns}
	
\begin{column}{0.55\textwidth}
\textbf{Broadband Adoption}
\begin{itemize}
	\item[(2)] High-skill intensity firms are more likely to adopt broadband internet
\end{itemize}
\vspace{5pt}
\textbf{What type of firms quickly adopt?} \\
Adopting firms are $\dots$
\begin{itemize}
	\item[(4)] relatively large or productive
	\item[(5)] more likely to deploy high-skill labor
	\item[(6)] more likely to use computers
\end{itemize}
\vspace{5pt}
\textbf{Endogeneous Technology Adoption:}
\begin{itemize}
	\item[$\Rightarrow$] Broadband Internet is adopted more likely if complementary factors are abundant
\end{itemize}


\end{column}

\begin{column}{0.45\textwidth}
	\centering
	\scriptsize{
	\begin{tabular}{l p{5cm}}
 	\underline{Table 3:} & Characterizing Adopting Firms \newline \textit{from baseline characteristics}
 	\end{tabular}}
 	
    \includegraphics[width=0.9\textwidth]{TableV.png}\qquad
    \tiny
    \textit{Low (High) $\rightarrow$ Skill intensity below (above) median}
    
\end{column}
	
\end{columns}
	
\end{frame}


%%%%%%% Conclusion %%%%%%%%%%%

\section{Conclusion}
\begin{frame}{Conclusion}

\textbf{Research Question} \\
\begin{itemize}
	\item[Q:] How does broadband adoption affect labor market outcomes and productivity at different skill levels, and what mechanisms drive this effect?
\end{itemize}

\textbf{Method} \\
\begin{itemize}
	\item[$\rightarrow$] Norwegian administrative data (2001–2007) on firms, workers, and broadband access.
	\item[$\rightarrow$] ITT Panel Data regression to estimate causal effects.
\end{itemize}

\textbf{Results} \\

\begin{itemize}
	\item[$\rightarrow$] Broadband benefits skilled (harms unskilled) workers.
	\item[$\rightarrow$] Adopting Firms have higher productivity, but gains are unevenly distributed.
	\item[$\Rightarrow$] \textbf{Key Takeaway:} Broadband adoption is an endogenous, skill-biased technological change, favoring skilled workers and reshaping labor markets.
\end{itemize}


\end{frame}

\end{document}









































